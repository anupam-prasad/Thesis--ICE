\documentclass[a4paper,10pt]{article}
\usepackage[utf8]{inputenc}
\usepackage{amsmath}
%opening
\title{ICE Model Continued}
\author{Anupam Prasad Vedurmudi}

\begin{document}

\maketitle

We denote the membrane displacements on either side by the
functions $u_{0/L}(r,\phi,t)$ where $0$ and $L$ correspond to the
ipsi- and contra-lateral membranes respectively. They satisfy
the boundary condition
\begin{equation}
 u_{0/L}(a,\phi,t)=0
\end{equation}
where $a$ is the membrane radius. We choose a convention in which
the displacements into the cavity are positive and outward displacements
negative. Thus, for a given displacement of the membranes, 
the cavity volume changes by
\begin{equation}
 \Delta V=-\int_S dS (u0+uL)
\end{equation}
where $dS=rdrd\phi$ is the area element in 2D polar coordinates and 
the integral is over the surface of the disk. Assuming the membrane
displacements are slow enough to consider the air inside the cavity
to be quasi-static, the corresponding change in pressure (linearized
in volume-change) is given by
\begin{equation}
 \Delta P \approx -\gamma \frac{P0}{V0}\Delta V
\end{equation}
$P0$ is the atmospheric pressure and $V_0$. The membrane equations of motion are then given by
\begin{equation}
 -\ddot{u}_{0/L}-2\alpha\dot{u}_{0/L}+c_M^2\Delta u_{0/L}=\frac{1}{\rho_M d}\left[p_{0/L}-\Delta P\right]
\end{equation}
Where $p_{0/L}$ is the sound pressure on the ipsi- and contra-lateral membranes respectively.	 We define 
a new set of variables, $u_+=u_L+u_0$ and $u_-=u_L-u_0$ and add and subtract the above equations to
get a new system in terms of these newly defined variables.

\begin{align}
 -\ddot{u}_+-2\alpha\dot{u}_++c_M^2\Delta u_+&=\frac{1}{\rho_M d}\left[p_+-2\Delta P\right]\\
 -\ddot{u}_--2\alpha\dot{u}_-+c_M^2\Delta u_-&=\frac{p_-}{\rho_M d}
\end{align}
Where, similar to the above definitions, $p_+=p_L+p_0$ and $p_-=p_L-p_0$. The second equation
can be solved exactly.
\end{document}
