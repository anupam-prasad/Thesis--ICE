\documentclass[a4paper,10pt]{article}
\usepackage[utf8]{inputenc}
\usepackage{amsmath}
\usepackage{graphicx}

%opening
\title{ICE Model }
\author{Anupam Prasad Vedurmudi}
\date{}
\begin{document}

\maketitle
In the ICE model, the mouth canal is assumed to be a cavity of volume $V_0$ 
with an arbitrary shape. Two rigidly clamped circular membranes are placed some distance
apart. We will later argue that if the cavity is small enough in some sense, 
the exact positions of the membranes and the shape of the cavity will not be 
important.

We define $(m,n)$ as the modes of the homogeneous cylindrical wave equation such that,
\begin{equation}\label{CylindricalHarmonic}
 (m,n)\equiv cos(m\phi)J_m\left(\mu_{mn}r\right)
\end{equation}
where $J_m$ is the Bessel function of the first kind of order $m$ and $J_m\left(\mu_{mn}a\right)=0$. 
Here, $\mu_{mn}a$ is the $n^{th}$ zero of $J_m$. 

\subsection{Vibration of a Circular Membrane}
The free, undamped vibrations of a clamped membrane of radius $a$ are governed by the 2D wave equation,
\begin{equation}\label{membrane1}
  \frac{1}{c_M^2}\frac{\partial^2u(r,\phi,t)}{\partial t^2}=\frac{1}{r}\frac{\partial}{\partial r}\left(r\frac{\partial u(r,\phi,t)}{\partial r}\right)+
 \frac{1}{r^2}\frac{\partial^2 u(r,\phi,t)}{\partial \phi^2}
\end{equation}
subject to the boundary condition, $u(a,\phi,t)=0$. Here, $c_M$ is the propagation velocity
of waves on the surface of the membrane. It is expressed in terms of the membrane density $\	$, thickness $d$ and tension $T$ as, 
$c_M=\sqrt{T/\rho_M d}$. The general solution is given by
\begin{equation}
 u(r,\phi,t)=\sum_{m,n}\left(M^+_{mn}e^{jm\phi}+M^-_{mn}e^{-im\phi}\right)J_m(\mu_{mn}r)e^{j\omega_{mn}t}
\end{equation}
Using the fact that the above expression has to satisfy periodic boundary conditions in $\phi$, the
above equation reduces to, 
\begin{equation}\label{membranegeneral}
 u(r,\phi,t)=\sum_{m,n}C_{mn}\cos(m\phi)J_m(\mu_{mn}r)e^{j\omega_{mn}t}
\end{equation}
$\mu_{mn}$ corresponds to the $n^{th}$ zero of $J_m$, i.e., $\mu_{mn}=k_{mn}/a$  and $\omega_{mn}=c_M\mu_{mn}$.
\subsubsection{Forced and Damped Vibration}
The forced vibrations of a damped circular membrane is governed by the equation,
\begin{align}\label{ForcedMembrane}
-\frac{1}{c_M^2}\frac{\partial^2u(r,\phi,t)}{\partial t^2}&+2\alpha\frac{\partial u(r,\phi,t)}{\partial t}\nonumber\\
&+\frac{1}{r}\frac{\partial}{\partial r}\left(r\frac{\partial u(r,\phi,t)}{\partial r}\right)+\frac{1}{r^2}\frac{\partial^2 u(r,\phi,t)}{\partial \phi^2}=\frac{1}{\rho_Md}pe^{j\omega t}
\end{align}
Here, the membrane is forced by a periodic pressure $pe^{j\omega t}$ uniformly over its surface and $\alpha$ is
the damping coefficient. We look for solutions of the form,
\begin{equation}\label{forcedmembranegeneral}
 u_{0/L}(r,\phi,t)=\sum_{m,n}C^{0/L}_{mn}\cos(m\phi)J_m(\mu_{mn}r)e^{j\omega t}
\end{equation}
where $0$ and $L$ refer to the ipsi- and contra-lateral membranes respectively.
\subsubsection{Conventions}

\subsection{Vibration of Coupled Unloaded Membranes}
We first treat the case in which we have two circular coupled by a cylindrical cavity. The forcing
on both tympani is given by,
\begin{align}
 &p_0=pe^{-jk\frac{L}{2}\sin\theta}\\
 &p_L=pe^{jk\frac{L}{2}\sin\theta}
\end{align}
This means that the sound pressure on both tympani has the same amplitude but differ in phase by, $kL\sin\theta$ where,
$\theta$ is the angle the sound source makes with the central axis of the head. We can assume that
the instantaneous pressure is constant over the tympanum as its dimensions are much smaller than 
the wavelength of the sound wave.

According to our convention, the positive displacement of both membranes is into the cavity. 
At a given instant of time, the change in cavity volume is given by,
\begin{align}
 \Delta V&=-\int dS \sum_{m,n}\left(C^0_{mn}+C^L_{mn}\right)\cos m\phi J_m(\mu_{mn}r)e^{j\omega t}\nonumber\\
 &=-\sum_{m,n}\left(C^0_{mn}+C^L_{mn}\right)L_{mn}e^{j\omega t}\nonumber\\
 &=-\sum_{n}\left(C^0_{0n}+C^L_{0n}\right)L_{0n}e^{j\omega t}\label{CavitydV}
\end{align}
where we have made the definition, $L_{mn}=\int dS\cos m\phi J_m(\mu_{mn}r)$. The final step is
due to the fact that $L_{mn}=0$ for $m\neq 0$.

In the next step, we assume that the the gas is in quasi-static equilibrium. This means that the pressure inside the cavity readjusts
at time scales much smaller than that of the sound wave (i.e. $\frac{2\pi}{\omega}$) and therefore at
a given instant, the pressure can be assumed to be uniform over the cavity. Using the equation of state for a reversible adiabatic process we can
get the change in pressure corresponding to the above change in volume,
\begin{align}
 P_0V_0^\gamma&=(P_0+\Delta P)(V_0+\Delta V)^\gamma\nonumber\\
 \Rightarrow \Delta P&=-\gamma\frac{P_0}{V_0}\Delta V+\mathcal{O}(\Delta V^2)
\end{align}
$P_0$ is the equilibrium atmospheric pressure. In the second step we express the change
in pressure upto first order in $\Delta V$.

Now, we substitute \eqref{forcedmembranegeneral} on the LHS of \eqref{ForcedMembrane} and the expressions
for sound pressure and $\Delta p$ on the RHS. This gives us,
\begin{equation}
 \rho_Md\sum_{m,n}\left(\omega^2-2j\alpha\omega-\omega^2_{mn}\right)C^{0/L}_{mn}\cos(m\phi)J_m(\mu_{mn}r)e^{j\omega t}=p_{0/L}-\Delta p
\end{equation}
We then express $\Delta p$ in terms of the membrane modes and integrate out the $(m,n)$ modes on either
side of the above equation to get,
\begin{align}
 \Omega_{0n}C^{0}_{0n}&=p_{0}K_{0n}-\gamma\frac{P_0}{V_0}K_{0n}\sum_{l}\left(C^0_{0l}+C^L_{0l}\right)L_{0l}\label{ipsimodes}\\
 \Omega_{0n}C^{L}_{0n}&=p_{L}K_{0n}-\gamma\frac{P_0}{V_0}K_{0n}\sum_{l}\left(C^0_{0l}+C^L_{0l}\right)L_{0l}\label{contramodes}
\end{align}
where we've again made use of the fact that $L_{mn}=0$ for $m\neq 0$ and made the following definitions,
\begin{align}
 \Omega_{mn}&=\rho_Md\left(\omega^2-2j\alpha\omega-\omega^2_{mn}\right)\\
 K_{mn}&=\frac{L_{mn}}{\int dS\cos^2 m\phi J^2_m(\mu_{mn}r)}
\end{align}
In order to solve the above system of equations we first add and subtract \eqref{ipsimodes} and \eqref{contramodes}
to get a new set of equations,
\begin{align}
&C^+_n+2\gamma\frac{P_0}{V_0}\mathcal{M}_n\sum_{l}C^+_lL_{0l}=(p_L+p_0)\mathcal{M}_n\label{plusmodes}\\
&C^-_n=(p_L-p_0)\mathcal{M}_n\label{minusmodes}
\end{align} 
where we have defined, $\mathcal{M}_n=\frac{K_{0n}}{\Omega_{0n}}$ and the new variables,
\begin{align}
 C^+_n=C^L_{0n}+C^0_{0n}\\
 C^-_n=C^L_{0n}-C^0_{0n}
\end{align}
We see that for a given $n$, $C^-_n$ is exactly known. Equation \eqref{plusmodes} can be solved approximately by cutting off the infinite summation at some point.
This gives us a finite system of linear equations for the approximate coefficients $\widetilde{C}^+_n$. Written
in matrix form this gives us,
\begin{equation}\label{plusapprox}
 \left(\mathbf{I}+\mathbf{d\otimes e}\right)\widetilde{\mathbf{C}}^+=\mathbf{p}\\
\end{equation}
Here, $\mathbf{C}$ is the column vector representing the approximate coefficients, $\mathbf{I}$ is the
identity matrix and $\mathbf{p}$ is the column vector representing the R.H.S of \eqref{plusmodes}. $\otimes$ is
the Kronecker product of the two vectors $\mathbf{d}$ and $\mathbf{e}$. They are given by,
\begin{align}
 \mathbf{d}_n&=2\gamma\frac{P_0}{V_0}\mathcal{M}_n\\
 \mathbf{e}_n&=K_{0n}
\end{align}
i.e., $\mathbf{D}$ is a diagonal matrix. Finally, we express the approximate coefficients in the expansion
of the membrane modes as,
\begin{align}
 \widetilde{C}^0_n=(\widetilde{C}^+_n-C^-_n)/2 \label{ipsiapproxfinal}\\
 \widetilde{C}^L_n=(\widetilde{C}^+_n+C^-_n)/2 \label{contraapproxfinal}
\end{align}


\end{document}

