\documentclass[a4paper,10pt]{article}
\usepackage[utf8]{inputenc}
\usepackage{amsmath}
\usepackage{graphicx}

%opening
\title{Useful Coefficients}
\author{Anupam Prasad Vedurmudi}
\date{}
\begin{document}

\maketitle
Final expression for volume displacements of the membranes.
\begin{align}
 S_0=G_{ipsi}p_0+G_{contra}p_L\\
 S_L=G_{ipsi}p_L+G_{contra}p_0
\end{align}

Defining the other useful objects in terms of the membrane/cylinder properties
\begin{align}
 \frac{1}{\Lambda}&=\sum_{m,n}\frac{\left(\int dS f_{mn}(r,\phi)\right)^2}{\Omega_{mn}\int dS f^2_{mn}(r,\phi)}\\
 \Omega_{mn}&=\rho_m d \left(\omega^2-2j\alpha\omega-\omega^2_{mn}\right)\\
\end{align}
As we will see, the above definition of $\Lambda$ makes subsequent calculations simpler.

Easier to calculate the sum and difference of the membrane displacements.
\begin{align}
 S^+&=(S^L+S^0)=\frac{p_L+p_0}{\Lambda+\Gamma^+}\\
 S^-&=(S^L-S^0)=\frac{p_L-p_0}{\Lambda+\Gamma^-}
\end{align}
Where,
\begin{align}
 \Gamma^+=-\frac{\rho c\omega \cot \frac{kL}{2}}{\pi a^2_{cyl}}\\
 \Gamma^-=\frac{\rho c\omega \tan \frac{kL}{2}}{\pi a^2_{cyl}}
\end{align}
We therefore have,
\begin{align}
 G_{ipsi}&=\left(\frac{1}{\Lambda+\Gamma^+}+\frac{1}{\Lambda+\Gamma^-}\right)/2\\
 G_{contra}&=\left(\frac{1}{\Lambda+\Gamma^+}-\frac{1}{\Lambda+\Gamma^-}\right)/2
\end{align}
It is also convenient to define,
\begin{align}
 \frac{G_{contra}}{G_{ipsi}}&=\frac{\frac{\rho c\omega \csc kL}{\pi a^2_{cyl}}}{\Lambda-\frac{\rho c\omega \cot kL}{\pi a^2_{cyl}}}\\
			     &=\frac{1}{\eta\sin kL- \cos kL}
\end{align}
Where we've also defined,
\begin{equation}
 \eta=\frac{\pi a^2_{cyl}\Lambda}{\rho c\omega}
\end{equation}

The pressure coefficients,
\begin{align}
 A&=-\frac{1}{\eta \sin kL}\left(S^0e^{-jkL}+S^L\right)\\
 B&=-\frac{1}{\eta \sin kL}\left(S^0e^{jkL}+S^L\right)
\end{align}







\end{document}


