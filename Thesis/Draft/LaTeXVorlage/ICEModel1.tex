\chapter{The ICE Model}

Our goal is to model the
middle-ear of the vertebrates in question in the simplest possible way while ensuring an accurately replication of its main properties. 
The main components of such a 
system are the mouth-cavity, the two tympani and the two extracollumellar
footplates (one on each tympanum). In general, the shape of the mouth-cavity is highly irregular and therefore
not conducive to an analytical treatment. Moreover, the system corresponds to a pair of second-order PDE's with
moving boundaries. For this reason we will need to make further approximations for the sake of expediency.

\section{Description}
In the earlier treatment of the ICE model, the mouth canal is modelled as a simple cylinder closed at 
both ends by rigidly clamped (baffled) circular membranes. As shown in \cite{vossenthesis}, The length of the cylinder was chosen to be equal
to the interaural distance. The advantage of using a cylindrical cavity model for the mouth cavity is that the pressure
distribution inside the cavity is easy to calculate - something that is even more important at higher frequencies
as the pressure distribution inside the cavity is highly non-uniform. 

The problem with this description is the fact that the volume of the model's cavity 
is an order of magnitude smaller than that of the mouth-cavity in the corresponding animal with similar
tympani and a the same interaural distance. In general, a smaller volume results in a stronger coupling - both in terms of an increased iTD and an increased
iLD. For this reason, the earlier model overestimates the iTDs at low frequencies and the iLDs at
high frequencies respectively.

\section{Sound Input}

\section{Internal Cavity}
We assume that the air inside the cavity obeys linear acoustics (briefly discussed in \ref{acousticappendix}). The pressure distribution inside the cavity
is therefore given by the $3$D acoustic wave-equation in cylindrical polar coordinates,	
\begin{equation}\label{pressurewaveeqn}
 \frac{1}{c^2}\frac{\partial^2 p(x,r,\phi,t)}{\partial t^2}=\frac{1}{r}\frac{\partial}{\partial r}\left(r\frac{\partial p(x,r,\phi,t)}{\partial r}\right)
 +\frac{1}{r^2}\frac{\partial p(x,r,\phi,t)}{\partial \phi^2}+\frac{\partial p(x,r,\phi,t)}{\partial x^2}
\end{equation}
where $c$ is the sound propagation velocity. The complete solution must take into account the boundary conditions at and within the cavity walls and the
ones at the air-membrane interface. We also note that the above equation implies that the animal's mouth is closed, which is typical for a waiting
animal. In order to solve \eqref{pressurewaveeqn} for a particular frequency $f$ (angular frequency $\omega=2\pi f$), we use the following
separation ansatz
\begin{equation}\label{pseparationansatz}
  p(x,r,\phi,t)=f(x)g(r)h(\phi)e^{j\omega t}
\end{equation}
which after substitution into the acoustic wave-equation leads to,
\begin{equation}\label{pseparationansatz2}
\begin{split}
 k^2f(x)g(r)h(\phi)+&f(x)h(\phi)\left[\frac{\partial^2 g(r)}{\partial r^2} + \frac{1}{r}\frac{\partial g(r)}{\partial r}\right] \\
 &+f(x)g(r)\frac{1}{r^2}\frac{\partial h(\phi)}{\partial \phi}+\frac{\partial^2 f(x)}{\partial x^2}=0
\end{split}
\end{equation}
with $k:=\omega/c$. This yields the following set of equations (ODEs):
\begin{align}
 \frac{d^2 f(x)}{dx^2}+\zeta^2f(x)&=0\\
 \frac{d^2 h(\phi)}{d\phi^2}+q^2h(\phi	)&=0\\
 \frac{\partial^2 g(r)}{\partial r^2} + \frac{1}{r}\frac{\partial g(r)}{\partial r}+\left[(\displaystyle\underbrace{k^2-\zeta^2}_{=:\nu^2})-\frac{q^2}{r^2}\right]g(r)&=0\label{besselequation1}
\end{align}
with separation constants $q$ and $\zeta$. The last equation is the Bessel differential equation \cite[p.~313]{copsonbessel} and its general solution is given by,
\begin{equation}
 g(r)=C_{qs}J_q(\nu r)+D_{qs}Y_q(\nu r).
\end{equation}
$J_q$ and $Y_q$ are the order-$q$ Bessel functions of the first and second kind respectively. The Bessel function of the second kind can be ignored as it diverges at $r=0$.
The solutions to the separated equations are therefore given by,
\begin{equation}
 f(x)=e^{\pm \zeta x},\ h(\phi)=e^{\pm j\phi},\text{and}\ g(r)=J_q(\nu r)
\end{equation}
with a specific solution to \eqref{pressurewaveeqn} given by,
\begin{equation}\label{specificpressure1}
 p(x,r,\phi;t)=\left[(A^+e^{jq\phi}+A^-e^{-jq\phi})e^{j\zeta x}+(B^+e^{jq\phi}+B^-e^{-jq\phi})e^{-j\zeta x}\right]J_q(\nu r)e^{j\omega t}.
\end{equation}
The coefficients $A^\pm$, $B^\pm$, $q$, $\zeta$ and $\nu$ will be subsequently determined by the boundary conditions.
\subsubsection{Pressure Boundary Conditions}
There are three sets of boundary conditions - 
\begin{itemize}
 \item Continuity and smoothness in $\phi$ which is equivalent to $h(0)=h(2\pi)$ and $h^\prime(0)=h^\prime(2\pi)$ where, $h^\prime=dh/d\phi$.
 \item Vanishing of the normal derivative at the cavity walls - $g^\prime(a_{cyl})=0$ ($a_{cyl}$ is the radius of the cylinder).
 \item Equating the membrane velocity to the air velocity at the membrane boundaries (to be discussed in the next section).
\end{itemize}
The first set of requirements is obvious. This reduces \eqref{specificpressure1} to 
\begin{equation}\label{specificpressure2}
 p(x,r,\phi;t)=\left[Ae^{j\zeta x}+Be^{-j\zeta x}\right]\cos q\phi J_q(\nu r)e^{j\omega t}.
\end{equation}
With $q$ constrained to be an integer.

The second and third are a result of the so called ``no-penetration'' boundary-condition
of fluid-mechanics. It arises from the from the fact that the cavity wall is an impermeable boundary. This translates 
into the requirement that the normal fluid-particle velocity should vanish (\cite[p.~111]{pozrikidisFluid}).
The fluid-particle velocity ($\mathbf{v}$) is related to the pressure by, 
\begin{equation}\label{pressurevelocityrelation}
 -\rho\frac{\partial \mathbf{v}}{\partial t}=\nabla p
\end{equation}
At the cylindrical cavity wall, the normal velocity is in the radial direction.
Substituting the expression for pressure in \eqref{specificpressure1} in the above equation leads to
a Neumann boundary condition for the pressure,
\begin{align}\label{radialnopenetration}
 v_r=-\left.\frac{1}{j\rho\omega}\frac{\partial p(x,r,\phi;t)}{\partial r}\right\vert_{r=a}&=0\nonumber\\
    \Rightarrow\left.\frac{\partial J_q (\nu r)}{\partial r}\right\vert_{r=a}&=0
\end{align}
This constrains $\nu$ to a discrete set of values which correspond to the local minima and maxima
of $J_q$. We can therefore index $\nu$ by $q$ and $s=0,1,2,3,\ldots$ with $\nu_{qs}=z_{qs}/a$:
$z_{qs}$ being the $s^{th}$ extremum of the order-$q$ Bessel function of the first kind.
This results in a discrete set of modes that satisfy \eqref{pressurewaveeqn} which are given
by,
\begin{equation}\label{specificpressure3}
 p_{qs}(x,r,\phi;t)=\left[A_{qs}e^{j\zeta_{qs}x}+B_q{s}e^{-j\zeta_{qs}x}\right]\cos q\phi J_q(\nu_{qs} r)e^{j\omega t}.
\end{equation}
where we have added the subscripts $q$ and $s$ to $\zeta$. Effectively, the
modes are 3D waves propagating with wave numbers $\zeta_{qs}$ in the $x$-direction and $\nu_{qs}$ in the radial
direction. The first of these modes (corresponding to $q=0,s=0$) is of particular importance. Since the first
maximum of $J_0$ occurs at $r=0$, we have $\nu_{00}=0$. This leads to the first mode being a plane-wave which
is constant in $r$ and $\phi$ and only varies in $x$. 

A very useful property of the above modes is their orthgonality, i.e.
\begin{equation}\label{pressureorthogonality}
 \int_\Omega dVp_{q_1s_1}p_{q_2s_2}=0,\ if\ q_1\neq q_2\ or\ s_1\neq s_2
\end{equation}
the integral is over the volume of the cylinder. This is a consequence of the fact that for a given $q$,
\begin{equation}\label{besselorthogonality}
 \int rdrJ_q(\nu_{qs_1}r)J_q(\nu_{qs_2}r)=0,\ if\ s_1\neq s_2
\end{equation}

We can therefore write the general solution to \eqref{pressurewaveeqn} as a linear combination of the orthogonal modes given in \eqref{specificpressure3},
\begin{equation}\label{pressuregeneral1}
 p(x,r,\phi;t)=\displaystyle\sum^\infty_{q=0}\displaystyle\sum^\infty_{s=0}\left(A_{qs}e^{j\zeta_{qs}x}+B_{qs}e^{-j\zeta_{qs}x}\right)\cos(m\phi)J_q(\nu_{qs}r)e^{j\omega t}
\end{equation}
The remaining coefficients, $A_{qs}$ and $B_{qs}$, will be determined by equating the fluid-particle velocity to
the membrane velocity at both ends of the cylinder. To do so, we will first need to find an expression
for the membrane vibrations and subsequently make use of some simplifying approximations.

\section{Vibration of the Membrane}
As a preliminary exercise, we will first derive expressions for the free and force-driven
vibrations of a circular membrane. We will then use our results to move on to the sectoral membrane 
which corresponds to the tympanum loaded by the extracollumella. Physically, this means we have
assumed the extracollumella to have infinite mass. 
\subsection{Circular Membrane}
The equation of motion for the vibration of a rigidly clamped circular membrane of radius $a$ solves for the membrane displacement $u$ at
a point $(r,\phi)$ with $r<a$ and $0<\phi<2\pi$. It is given by,
\begin{equation}\label{membraneequation1}
 -\frac{\partial^2 u(r,\phi,t)}{\partial t^2}-2\alpha\frac{\partial u(r,\phi,t)}{\partial t}+c^2_m\left[\frac{1}{r}\frac{\partial}{\partial r}\left(r\frac{\partial u(r,\phi,t)}{\partial r}\right)
 +\frac{1}{r^2}\frac{\partial u(r,\phi,t)}{\partial \phi^2}\right]=\frac{1}{\rho_m d}\Psi(r,\phi,t)
\end{equation}
subject to the boundary condition $u(r,\phi,t)|_{r=a}=0$.. We've defined the following membrane material properties,
\begin{itemize}
 \item $c_m$ - propagation speed of vibrations.
 \item $\alpha$ - the damping coefficient.
 \item $\rho_m$ - density.
 \item $d$ - thickness.
\end{itemize}
$\Psi(r,\phi;t)$ is the pressure on the membrane surface at $(r,\phi)$. In our discussion we are only concerned with periodic and uniform 
pressure acting on the membrane surface. This is justified by the fact that for typical hearing ranges of these animals, the wavelength
of sound is much greater than the membrane size and any spatial variation can be neglected.
\subsubsection*{Free Undamped Vibrations}
We first determine the eigenmodes of a circular membrane by solving \eqref{membraneequation1} for $\alpha=0,\ \Psi=0$. To do this 
we make a separation ansatz just as we did in \eqref{pseparationansatz},
\begin{equation}\label{mseparationansatz}
 u(r,\phi,t)=f(r)g(\phi)h(t)
\end{equation}
This gives us the following set of equations
\begin{align}
 \frac{\partial^2 f(r)}{\partial r^2} + \frac{1}{r}\frac{\partial f(r)}{\partial r}+\left[\mu^2-\frac{m^2}{r^2}\right]f(r)&=0\label{besselequation2}\\
  \frac{d^2 g(\phi)}{d\phi^2}+m^2g(\phi)&=0\\
 \frac{d^2 h(t)}{dt^2}+c^2_m\mu^2h(t)&=0
\end{align}
with separation constants $\mu$ and $m$. The solution of the first of these equations should already be familiar to us from the previous section - $J_m(\mu r)$,
 the order-$m$ Bessel function of the first kind. The boundary conditions in $\phi$ direction remain the same resulting in,
\begin{equation}\label{specificmembrane1}
 u(r,\phi;t)=\left[M^+e^{jm\phi}+M^-e^{-jm\phi}\right] J_m(\mu r)e^{jc_m\mu t}
\end{equation}
Unlike in the case of the internal cavity, we require $u$ to vanish at the boundary so we have a Dirichlet boundary condition which
effectively requires: $J_m(\mu a)=0$. This constrains $\mu$ to a discrete set of values which correspond to the zeros of $J_m$. The eigenmodes of a 
the circular membrane are therefore given by,
\begin{equation}\label{membraneeigen}
 u_{mn}(r,\phi;t)=\left[M^+_{mn}e^{jm\phi}+M^-_{mn}e^{-jm\phi}\right] J_m(\mu_{mn} r)e^{j\omega_{mn} t}
\end{equation}
where $\mu_{mn}=z_{mn}/a$, $z_{mn}$ being the $n^{th}$ zero of $J_m$ and, $\omega_{mn}=c_m\mu_{mn}$ is
the eigenfrequency of the $(m,n)$ eigenmode. At this point $m$ can take any positive real value -- a fact that will
help us solve the sectoral membrane problem. However, in the case of a full circular membrane -- as in the case
of the pressure inside a cylindrical cavity -- requirements of continuity and smoothness in $\phi$ reduce \eqref{membraneeigen}
to,
\begin{equation}\label{circularmembraneeigen}
 u_{mn}(r,\phi;t)=\cos m\phi J_m(\mu_{mn} r)e^{j\omega_{mn} t}
\end{equation}
with $m=0,1,2,\ldots$ with the $(m,n)$ eigenmodes forming an orthogonal set.
\subsubsection*{Forced Vibrations}
For a periodically driven membrane, there are two components of the full solution for forced vibrations. The first of these is the steady state solution which
oscillates with the same frequency as the input and does not depend on the initial conditions - $u_{ss}$. The second of these is the transient solution that depends
on the initial conditions but not on the driving pressure - $u_t$. 

The steady state solution is expressed as a linear combination of the spatial part
of the above eigenmodes and is given by,
\begin{equation}\label{membraness1}
 u_{ss}(r,\phi ;t)=\displaystyle\sum^\infty_{m=0}\sum^\infty_{n=1} C_{mn}\cos m\phi J_m(\mu_{mn} r)e^{j\omega t}.
\end{equation}
Substituting this expression in \eqref{membraneequation1} gives,
\begin{align}
 &\displaystyle\sum^\infty_{m=0}\sum^\infty_{n=1} \Omega_{mn}C_{mn}\cos m\phi J_m(\mu_{mn} r)e^{j\omega t}=pe^{j\omega t}\label{membraness2}\\
 &\Omega_{mn}=\rho_m d (\omega^2-2j\alpha\omega-\omega^2_mn)\label{omegafirstdef}
\end{align}


\subsection{Sectoral Membrane}