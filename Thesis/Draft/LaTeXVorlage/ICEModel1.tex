\chapter{The ICE Model}

Our goal is to model the
middle-ear of the vertebrates in question in the simplest possible way while ensuring an accurately replication of its main properties. 
The main components of such a 
system are the mouth-cavity, the two tympani and the two extracollumellar
footplates (one on each tympanum). In general, the shape of the mouth-cavity is highly irregular and therefore
not conducive to an analytical treatment. Moreover, the system corresponds to a pair of second-order PDE's with
moving boundaries. For this reason we will need to make further approximations for the sake of expediency.

\section{Description}
In the earlier treatment of the ICE model, the mouth canal is modelled as a simple cylinder closed at 
both ends by rigidly clamped (baffled) circular membranes. As shown in \ref{oldICE}, The length of the cylinder was chosen to be equal
to the interaural distance. The advantage of using a cylindrical cavity model for the mouth cavity is that the pressure
distribution inside the cavity is easy to calculate - something that is even more important at higher frequencies
as the pressure distribution inside the cavity is highly non-uniform. 

The problem with this description is the fact that the volume of the model's cavity 
is an order of magnitude smaller than that of the mouth-cavity in the corresponding animal with similar
tympani and a the same interaural distance. In general, a smaller volume results in a stronger coupling - both in terms of an increased iTD and an increased
iLD. For this reason, the earlier model overestimates the iTDs at low frequencies and the iLDs at
high frequencies respectively.

\section{Vibration of the Membrane}

\subsection{Circular Membrane}
\subsection{Sectoral Membrane}