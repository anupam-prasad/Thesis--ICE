\chapter{Summary and Conclusion}
We started our study in Chapter \ref{modelchapter} by constructing a basic model that describes the motion of eardrums in animals
with coupled ears. The mouth cavity of the animal was modelled as an air-filled cylindrical tube
of radius $a_{cyl}$ with circular openings on either side of radius $a_{tymp}$ that are closed by
tympanic membranes of the same radius. This is one of the points where we deviate from the previous treatment
of the problem \cite{vossenjasa} as we have kept the volume of the cavity $V_{cav}$ constant and equal to
typical values for the species in nature. The sound inputs to both ears took the form of pressure waves
with a phase shift which depended on the frequency, direction of the source and head width and shape.

The second difference of our model from the previous treatment was the modelling of the membrane
as a circular sector of angle $2\pi-2\beta$. The remaining sector of angle $2\beta$ represented the 
the extracolumella - the extension whose function is to serve as a transducer of
membrane vibrations. We had modelled the extracolumella as being a stationary object and its
effect on the membranes was given through additional boundary conditions at $\phi=\pm\beta$. The variation of air-pressure
inside the tube was described by the $3D$ wave equation with no-penentration boundary conditions at the wall. In the previous
treatment it was assumed that the pressure modes inside the cavity are identical to the circular modes on the membrane. 

We then proceeded to find a solution to the full problem viz., the vibration of the membranes coupled through the cylindrical
cavity. This required the application of a new set of boundary conditions at the membrane-air interface. These conditions were
equivalent to the no-penetration boundary condition and required setting the membrane velocity to be equal to the velocity of
air inside the cavity. The complexity of these boundary conditions required an approximation. We expanded the membrane 
velocity in terms of the pressure modes and used the zeroeth order of the expansion. Thus we effectively approximated the membrane
by a piston oscillating with an amplitude equal to the integral of the membrane displacement. The approximation of the boundary
condition simplified the expression of the pressure by approximating it as a plane wave propagating along the $x$-axis. In the
frequency ranges of interest to us this turned out to be a reasonable assumption. At the end of the chapter we obtained expressions
for the membrane displacements in terms of the input pressure inputs to both ears.

In Chapter \ref{modelanalysis} we evalueated the results of our model - specifically its direction and frequency. We started
by comparing the total membrane displacements $S^{0/L}$ (the integrals of the membrane displacement) with the experimentally 
measured values of displacement at the tip of the extracolumella. 
We then went on to introduce the main directional cues in terms of the ratio $S^{0}$ and $S^{L}$. Namely, the Internal Level Difference
(iLD) which was defined as the decibel of the absolute value of this ratio and the Internal Time Difference (iTD) defined as the complex
argument divided by the angular frequency $\omega$.


\section{Open Questions}
The main advantage of the ICE-model is to explain the frequency and direction dependence of the
hearing cues.