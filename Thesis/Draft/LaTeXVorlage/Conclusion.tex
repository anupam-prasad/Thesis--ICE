\chapter{Summary and Discussion}
We started our study in Chapter \ref{modelchapter} by constructing a basic model that describes the motion of eardrums in animals
with coupled ears. The mouth cavity of the animal was modelled as an air-filled cylindrical tube
of radius $a_{\mathrm{\mathrm{cyl}}}$ with circular openings on either side of radius $a_{\mathrm{tymp}}$ that are closed by
tympanic membranes of the same radius. This is one of the points where we deviate from the previous treatment
of the problem \cite{vossenjasa} as we have kept the volume of the cavity $V_{\mathrm{cav}}$ constant and equal to
typical values for the species in nature. The sound inputs to both ears took the form of pressure waves
with a phase shift which depended on the frequency, direction of the source and head width and shape.

The second difference of our model from the previous treatment was the modelling of the membrane
as a circular sector of angle $2\pi-2\beta$. The remaining sector of angle $2\beta$ represented the 
the extracolumella - the extension whose function is to serve as a transducer of
membrane vibrations. We had modelled the extracolumella as being a stationary object and its
effect on the membranes was given through additional boundary conditions at $\phi=\pm\beta$. The variation of air-pressure
inside the tube was described by the $3D$ wave equation with no-penetration boundary conditions at the wall. In the previous
treatment it was assumed that the pressure modes inside the cavity are identical to the circular modes on the membrane. 

We then proceeded to find a solution to the full problem viz., the vibration of the membranes coupled through the cylindrical
cavity. This required the application of a new set of boundary conditions at the membrane-air interface. These conditions were
equivalent to the no-penetration boundary condition and required setting the membrane velocity to be equal to the velocity of
air inside the cavity. The complexity of these boundary conditions required an approximation. We expanded the membrane 
velocity in terms of the pressure modes and used the zeroeth order of the expansion. Thus we effectively approximated the membrane
by a piston oscillating with an amplitude equal to the integral of the membrane displacement. The approximation of the boundary
condition simplified the expression of the pressure by approximating it as a plane wave propagating along the $x$-axis. In the
frequency ranges of interest to us this turned out to be a reasonable assumption. At the end of the chapter we obtained expressions
for the membrane displacements in terms of the input pressure inputs to both ears.

In Chapter \ref{modelanalysis} we evaluated the results of our model - specifically its direction and frequency. We started
by comparing the total membrane displacements $S^{0/L}$ (the integrals of the membrane displacement) with the experimentally 
measured values of displacement at the tip of the extracolumella. 
For the membrane velocities, we found that they vary systematically with
direction and frequency and were in general higher in ipsilateral directions. The experimentally observed properties such
as the marked asymmetry, steepness across the midline and general frequency response were also reproduced. The shape of the membrane
velocities suggested that monoaural computation of membrane velocities would not serve well as directional cues.

We then went on to introduce the main directional cues in terms of the ratio $S^{0}$ and $S^{L}$. Namely, the Internal Level Difference
(iLD) which was defined as the decibel of the absolute value of this ratio and the Internal Time Difference (iTD) defined as its complex
argument divided by the angular frequency $\omega$. In order to stress on the universality of the model, the calculated values of these
 quantities were compared with experimentally measured ones for two gekkonids - the common house gecko.(\emph{Hemidactylus frenatus}) and the Tokay gecko
 (\emph{Gekko gecko}). The simultaneous frequency and direction dependence of the iLD was illustrated by
means of contour plots and were found to be consistent with experimental measurements - both in terms of qualitative spectral behaviour and
quantitative values thus justifying our choice of $S^{0/L}$ as measures of membrane displacement. We then introduced the requirements these cues must fulfil in order to effectively serve as
directional cues. For the iLDs it was found that these requirements are best satisfied at relatively higher frequencies while
 iTDs seem to be more effective at lower frequencies.
 
The model thus inherently contains frequency regimes for both directional cues. While in the case of mammals the use of ITD/ILD
is dictated by the size of the head in comparison to the wavelength, the transition in the ICE model is dictated my the membrane
properties like the fundamental eigenfrequency and damping. Certain parameters of the model were directly based on experimental
values - membrane density, radius and thickness, cavity volume, head width etc. The other membrane specific parameters such
as its first eigenfrequency and its damping coefficient are hard to experimentally determine. In Sec. \ref{parameterestimation} we therefore presented
a possible method to estimate these parameters from observations. We ended the chapter by illustrating the vibrations patterns of one of the ICE-model
membranes with the opposite one blocked. These patterns were compared and found to be in good agreement with the patterns experimentally measured for the Tokay gecko by Manley \cite{manleygecko1}.
\section{Biological Relevance}
In this thesis we have presented a general model for ears coupled through a large air cavity. Although we
have restricted the comparison of our results with membrane vibration data for gekkonids from Dalsgaard and Manley (\cite{dalsgaardmanley1},\cite{dalsgaardmanley2})
and Dalsgaard \emph{et al} \cite{dalsgaardtangcarr}, our model readily describes the coupling between eardrums across a range of gekkonid species by substituting
the appropriate parameters for cavity volume ($V_{\mathrm{cav}}$), head width, tympanum area etc. The model can also be adapted to other animals with ICE such as
birds and the main modification to be made would be to the transducer (the extracolumella in our case). For example, the extracolumella in birds is generally
centrally attached to the eardrum \cite{millsavianmiddleear} resulting in a different set of membrane eigenmodes. 

The agreement of our calculations with
experimental results suggests that the exact shape of the cavity does not have an effect as significant as $V_{\mathrm{cav}}$ on the hearing cues.  The previously
unaccounted for effect of diffraction on the input phase difference was also seen to have a significant effect on both the iTDs and iLDs.
The ICE model also places natural limits on iTD and iLD based hearing dependent on the fundamental membrane eigenfrequency. As we have seen in Sec. \ref{itdsubsection},
the transition frequency from iTD based localization to iLD based localization is close to the membrane eigenfrequency. For this reason it isn't unreasonable to expect that animals with typically
higher hearing ranges will tend to have higher membrane eigenfrequencies and vice versa. In contrast with mammals, in whom this transition is determined by their size, the transition
between the two regimes is entirely a consequence of the physics of the ICE model.

By using a cylindrical air cavity as opposed to a circuit model for the coupled system (Zhang \emph{et al} \cite{zhanghallam}), we have been able to account for the
high frequency behaviour of the system which is a result of significant pressure amplitude differences between the ears. This also leads to the additional possibility of using the ICE model to explain hearing in
animals such as crocodilia. As shown by Vergne \emph{et al} \cite{vergnecrocodilia} and Higgs \emph{et al} \cite{higgscrocodilia} they seem to be able to hear well both above and underwater while
using the same neural processing pathways for both enviroments.
The speed of sound in water is four times that in air and as a result wavelengths for the same frequency are quadrupled as well. There is a possibility of switching to an ICE
like system to overcome this problem when underwater.

\section{Open Questions}
The main advantage of the ICE-model is to explain the frequency and direction dependence of the
hearing cues. In order to have a complete quantitative description of the ICE-model, we would also need to take
into account the motion of the extracolumella and the exact shape of realistic mouth cavities. The first of
these can be treated analytically up to a point. The main modification would be in the membrane boundary
conditions corresponding to the extracolumella. Instead of setting the displacement to zero, we would need to take into
account the mass of the extracolumella resulting in a new equation of motion. This equation would describe the motion
of the extracolumella driven by the membrane tension and the internal and external pressure difference. A further step
would be to incorporate the flection of the extracolumella at higher frequencies as described by Manley \cite{manleygecko1}. This would require a further 
understanding of its constituent cartilaginous material and would necessitate a numerical treatment.

Due the complex shape of a realistic mouth cavity, a full treatment is not conducive to an analytical treatment.
Numerical software like COMSOL can be used to treat reconstructions of mouth casts. Another possibility
would be to include the effect of the nostrils as either a new set of boundary conditions or a third input source. The eigenfrequencies
of the mouth cavity were studied by Vossen \cite[p.~39]{vossenthesis} but the influence of the holes corresponding to the membranes was neglected here. Additionally,
the neuronal basis of the computation of iTD cues is not fully understood and the applicability of the Jeffress model in their
case has also been called into question. Lastly, the response of the animal to sound stimuli in realistic environments viz. in the presence
of multiple obstacles could also be a topic for further study. This however necessitates further experimental analyses of sound localization in
vertebrates.