\chapter{The ICE Model}

Our goal is to model the
middle-ear of the vertebrates in question in the simplest possible way while ensuring an accurately replication of its main properties. 
The main components of such a 
system are the mouth-cavity, the two tympani and the two extracollumellar
footplates (one on each tympanum). In general, the shape of the mouth-cavity is highly irregular and therefore
not conducive to an analytical treatment. Moreover, the system corresponds to a pair of second-order PDE's with
moving boundaries. For this reason we will need to make further approximations for the sake of expediency.

\subsection{Description}
In the earlier treatment of the ICE model, the mouth canal is modelled as a simple cylinder closed at 
both ends by rigidly clamped (baffled) circular membranes. As shown in \ref{oldICE}, The length of the cylinder was chosen to be equal
to the interaural distance. The problem with this description is the fact that the volume of the model's cavity 
is an order of magnitude smaller than that of the mouth-cavity in a corresponding animal with similar
tympani and a the same interaural distance.

We define $(m,n)$ as the modes of the homogeneous cylindrical wave equation such that,
\begin{equation}\label{CylindricalHarmonic}
 (m,n)\equiv cos(m\phi)J_m\left(\mu_{mn}r\right)
\end{equation}
where $J_m$ is the Bessel function of the first kind of order $m$ and $J_m\left(\mu_{mn}a\right)=0$. 
Here, $\mu_{mn}a$ is the $n^{th}$ zero of $J_m$. 

\subsection{Internal Cavity}
The propagation of a pressure disturbance $p$ in the internal cavity is assumed to be defined by the following
equation
\begin{equation}\label{CavityCylindrical}
 \frac{1}{c^2}\frac{\partial^2p(x,r,\phi,t)}{\partial t^2}=\frac{1}{r}\frac{\partial}{\partial r}\left(r\frac{\partial p(x,r,\phi,t)}{\partial r}\right)+
 \frac{1}{r^2}\frac{\partial^2 p(x,r,\phi,t)}{\partial \phi^2}+\frac{\partial^2 p(x,r,\phi,t)}{\partial x^2}
\end{equation}
i.e., the wave equation in cylindrical coordinates. $c$ is the speed of sound in air. We can use a separation
ansatz, $p(x,r,\phi,t)=f(x)g(r)h(\phi)e^{j\omega t}$ to find a particular solution
to this equation; this solution is given by,

\begin{align}\label{CavityCylSpec}
p(x,r,\phi,t)=&\left[\left(A^+_{qs}e^{j\zeta_{qs}x} + B^+_{qs}e^{-i\zeta_{qs}x}\right)e^{jq\phi}\right.\nonumber\\
                    &+\left.\left(A^-_{qs}e^{j\zeta_{qs}x}+B^-_{qs}e^{-i\zeta_{qs}x}\right)e^{-iq\phi}\right] J_p(\nu_{qs}r)e^{j\omega t}
\end{align}
Here, $g(r)=J_p(\nu_{qs}r)$ is the order $p$ Bessel function of the first kind wich satisfies the following
second order linear ODE
\begin{equation}\label{besseleqn}
 \frac{\partial^2 g(r)}{\partial r^2}+\frac{1}{r}\frac{\partial g(r)}{\partial r}+\left(\nu^2_{qs}-\frac{q^2}{r^2}\right)g(r)=0
\end{equation}
where, 
\begin{align}
&k=\omega/c\\
&\zeta^2_{qs}=k^2-\nu^2_{qs}\label{wavenumbers}
\end{align}
The velocity of air (physically, the velocity of the fluid particle) in the x-direction is given by
\begin{equation}\label{airvelocity}
 \rho\frac{\partial v_x}{\partial t}=-\nabla_x p
\end{equation}
 where $\rho$ is the density of the air inside the cavity. Since \eqref{CavityCylindrical} is 
a second order PDE in each of its variables, we require two boundary conditions for each of them 
to completely solve it, i.e. to determine all the coefficients in \eqref{CavityCylSpec}. We require
the pressure and its derivative to be periodic in $\phi$, 
\begin{align}
 p(x,r,\phi,t)&=p(x,r,\phi+2\pi,t)\label{phboundary1}\\
 \partial_\phi p(x,r,\phi,t)&=\partial_\phi p(x,r,\phi+2\pi,t)\label{phboundary2}
\end{align}
As a result, $q$ is required to be an integer and \eqref{CavityCylSpec} reduces to
\begin{equation}\label{CavityCylSpec2}
 p(x,r,\phi,t)=\left[A_{qs}e^{j\zeta_{qs}x}+B_{qs}e^{-i\zeta_{qs}x}\right]\cos(q\phi) J_q(\nu_{qs}r)e^{j\omega t}
\end{equation}
Finally, we require that the velocity of the fluid particle normal to the cylindrical boundary at
$r=a$ disappears. This is due to the requirement that the boundary is solid and the fluid does not
penetrate it. This means that,
\begin{equation}\label{normalboundary}
 \left.\frac{\partial J_q(\nu_{qs}r)}{\partial r}\right|_{r=a}=0
\end{equation}
As a result $\nu_{qs}=\widetilde{k}_{qs}/a$, i.e. $\nu_{qs}$ corresponds to the $s^{th}$ zero of $J^\prime_q$. The general solution is given by linear combinations of \eqref{CavityCylSpec2}. 
We also note that there exists a plane wave solution to \eqref{CavityCylindrical} which corresponds to
$\nu_{00}=0$. This is given by
\begin{equation}\label{CylinderPlaneWave}
 p(x,r,\phi;t)=\left[A_{00}e^{jkx}+B_{00}e^{-jkx}\right]e^{j\omega t}
\end{equation}

\subsection{Vibration of Coupled Unloaded Membranes}
We first treat the case in which we have two circular coupled by a cylindrical cavity. The forcing
on both tympani is given by

\begin{align}
 &p_0=pe^{-jk\frac{L}{2}\sin\theta}\\
 &p_L=pe^{jk\frac{L}{2}\sin\theta}
\end{align}
In this example we attempt to model the case of the animal to a free field stimulus. This means
that the sound on both tympani has the same amplitude but differ in phase by, $kL\sin\theta$ where,
$\theta$ is the angle the sound source makes with the central axis of the head. We can assume that
the instantaneous pressure is constant over the tympanum as its dimensions are much smaller than 
the wavelength of the sound wave.
The pressure inside the cavity is given by,
\begin{equation}\label{CylCavityPressure}
 p(x,r,\phi,t)=\sum_{q,s}\left[A_{qs}e^{j\zeta_{qs}x}+B_{qs}e^{-j\zeta_{qs}x}\right]\cos(q\phi) J_p(\nu_{qs}r)e^{j\omega t}
\end{equation}
The equation of motion of the membranes gives us,
\begin{align}
 \sum_{m,n}\left(\omega^2-2j\alpha\omega-\omega^2_{mn}\right)C^0_{mn}\cos(m\phi)&J_m(\mu_{mn}r)\nonumber\\
 &=\left[p_0-p(0,r,\phi,t)\right]/(\rho_M d)\label{membraneEOM1a}\\
 \sum_{m,n}\left(\omega^2-2j\alpha\omega-\omega^2_{mn}\right)C^L_{mn}\cos(m\phi)&J_m(\mu_{mn}r)&\nonumber\\
 &=\left[p_L-p(L,r,\phi,t)\right]/(\rho_M d)\label{membraneEOM1b}
\end{align}
where as usual, $0$ and $L$ refer to the ipsi- and contralateral membranes respectively. We then equate the
the velocity of air given in \eqref{airvelocity} to the velocity of the membrane surface at that point. 
\begin{align}
 \rho\omega^2\sum_{m,n}C^0_{mn}\cos(&m\phi)J_m(\mu_{mn}r)=\sum_{q,s}j\zeta_{qs}\left[A_{qs}-B_{qs}\right]\cos(q\phi) J_p(\nu_{qs}r)\label{vboundary1a}\\
 \rho\omega^2\sum_{m,n}C^L_{mn}\cos(&m\phi)J_m(\mu_{mn}r)\nonumber\\
 &=-\sum_{q,s}j\zeta_{qs}\left[A_{qs}e^{j\zeta_{qs}L}-B_{qs}e^{-j\zeta_{qs}L}\right]\cos(q\phi) J_p(\nu_{qs}r)\label{vboundary1b}
\end{align}
This is equivalent \eqref{normalboundary} as we've set the relative velocity of the air normal to the 
membrane surface to zero. We note that we have also used the direction conventions mentioned in the previous section.

First we use \eqref{vboundary1a} and \eqref{vboundary1b} to express the cavity coefficients in terms
of the membrane coefficients. This gives us,
\begin{align}
 A_{ms}&=\frac{\rho\omega^2}{2\zeta_{ms}\sin(\zeta_{ms}L)}\sum_{l}\left[C^0_{ml}e^{-j\zeta_{ms}L}+C^L_{ml}\right]\lambda_{msl}\\
 B_{ms}&=\frac{\rho\omega^2}{2\zeta_{ms}\sin(\zeta_{ms}L)}\sum_{l}\left[C^0_{ml}e^{j\zeta_{ms}L}+C^L_{ml}\right]\lambda_{msl}
\end{align}
where we've introduced,
\begin{equation*}
 \lambda_{msl}=\frac{\int^a_0 rdr J_m(\nu_{ms}r)J(\mu_{ml}r)}{\int^a_0 rdrJ^2_m(\nu_{ms}r)}
\end{equation*}
We have also made use of the fact that $\int_0^{2\pi}\cos(m\phi)\cos(k\phi)=0$ for $m\neq k$. This tells
us that the $J_m$ modes on the membrane only couple to the $J_m$ modes in the cavity for a given $m$ and vice
versa. Also, in the above summations, $l$ ranges from $1$ to $\infty$ while $n$ can also be $0$.
We now substitute the above expressions into \eqref{membraneEOM1a} and \eqref{membraneEOM1b} to get,
\begin{align}
 \sum_{m,n}\Omega_{mn}C^0_{mn}\cos(m\phi)&J_m(\mu_{mn}r)\nonumber\\
 &=p_0-\sum_{q,s}\sum_{l}\left[\Lambda_{qs}C^0_{ql}+\Gamma_{qs}C^L_{ql}\right]\lambda_{qsl}f_{qs}(r,\phi)\label{membraneEOM2a}\\
 \sum_{m,n}\Omega_{mn}C^L_{mn}\cos(m\phi)&J_m(\mu_{mn}r)\nonumber\\
 &=p_L-\sum_{q,s}\sum_{l}\left[\Gamma_{qs}C^0_{ql}+\Lambda_{qs}C^L_{ql}\right]\lambda_{qsl}f_{qs}(r,\phi)\label{membraneEOM2b}
\end{align}
where we have defined
\begin{align}
 &\Omega_{mn}=\rho_M d\left(\omega^2-2j\alpha\omega-\omega^2_{mn}\right),\\
 &\Lambda_{qs}=\rho\omega^2 \frac{\cot(\zeta_{qs}L)}{\zeta_{qs}},\\
 &\Gamma_{qs}=\frac{\rho\omega^2}{\zeta_{qs}\sin(\zeta_{qs}L)},\\
 &f_{qs}(r,\phi)=\cos(q\phi)J_q(\nu_{qs}r)
\end{align}
We now multiply both sides of \eqref{membraneEOM2a} and \eqref{membraneEOM2b} with the membrane modes and
integrate over the circle to get,
\begin{align}
 \Omega_{mn}C^0_{mn}=p_0K_{mn}-\sum_{s,l}\left[\Lambda_{ms}C^0_{ml}+\Gamma_{ms}C^L_{ml}\right]\lambda_{msl}\widetilde{\lambda}_{msn}\\
 \Omega_{mn}C^L_{mn}=p_LK_{mn}-\sum_{s,l}\left[\Gamma_{ms}C^0_{ml}+\Lambda_{ms}C^L_{ml}\right]\lambda_{msl}\widetilde{\lambda}_{msn}
\end{align}
where, as usual, we have made the following definitions for aesthetic reasons,
\begin{align}
 &K_{mn}=\frac{\int^a_0 dS cos(m\phi)J_m(\mu_{mn}r)}{\int^a_0 dS cos^2(m\phi)J^2_m(\mu_{mn}r)}\\
 &\widetilde{\lambda}_{msn}=\frac{\int^a_0 rdr J_m(\nu_{ms}r)J(\mu_{mn}r)}{\int^a_0 rdrJ^2_m(\mu_{mn}r)}
\end{align}
In order to get a more symmetric looking expression, we can redefine $\lambda$ and $\widetilde{\lambda}$
in the following way,
\begin{align}
 &\lambda_{msl}\widetilde{\lambda}_{msn}=\widehat{\lambda}_{msl}\widehat{\lambda}_{msn}\\
 &\widehat{\lambda}_{msl}=\frac{\int^a_0 rdrJ_m(\nu_{ms}r)J_m(\mu_{ml}r)}{\sqrt{\int^a_0 rdrJ^2_m(\mu_{ml}r)\int^a_0 rdrJ^2_m(\nu_{ms}r)}}
\end{align}

We see that $I_{mn}$ vanishes for $m\geq1$. This means that we can ignore these modes. We, therefore neglect
the terms which have $m\neq 0$ to get,
\begin{align}
 \Omega_{0n}C^0_{0n}=p_0K_{0n}-\sum_{s,l}\left[\Lambda_{0s}C^0_{0l}+\Gamma_{0s}C^L_{0l}\right]\widehat{\lambda}_{0sl}\widehat{\lambda}_{0sn}\\
 \Omega_{0n}C^L_{0n}=p_LK_{0n}-\sum_{s,l}\left[\Gamma_{0s}C^0_{0l}+\Lambda_{0s}C^L_{0l}\right]\widehat{\lambda}_{0sl}\widehat{\lambda}_{0sn}
\end{align}
We now define a new set of variables, $C_n^+=C_{0n}^L+C_{0n}^0$ and $C_n^-=C_{0n}^L-C_{0n}^0$ and add and subtract the above
equations to get a system in terms of the newly defined variables,
\begin{align}
 \Omega_{0n}C^+_n+\sum_{s,l}\left[\Lambda_{0s}+\Gamma_{0s}\right]C^+_{l}\widehat{\lambda}_{0sl}\widehat{\lambda}_{0sn}=(p_L+p_0)K_{0n}\\
 \Omega_{0n}C^-_n+\sum_{s,l}\left[\Lambda_{0s}-\Gamma_{0s}\right]C^-_{l}\widehat{\lambda}_{0sl}\widehat{\lambda}_{0sn}=(p_L-p_0)K_{0n}
\end{align}

\subsection{Comparison}
We now want to compare our analysis with the previous one where the cavity was said to 
have an arbitrary shape. Since we reduced both systems to the same form, we can go ahead
and directly compare the respective coefficients.
